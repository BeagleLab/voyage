\documentclass{sig-alternate}

\usepackage{tikz}
\usetikzlibrary{arrows}
\usetikzlibrary{trees}
\usetikzlibrary{positioning}

\usepackage{array}
\usepackage{amstext}
\usepackage{mathtools}
\DeclarePairedDelimiter{\ceil}{\lceil}{\rceil}

\begin{document}

\title{Beagle - Navigating Academic Research}
\subtitle{}

\numberofauthors{1}

\author{
% You can go ahead and credit any number of authors here,
% e.g. one 'row of three' or two rows (consisting of one row of three
% and a second row of one, two or three).
%
% The command \alignauthor (no curly braces needed) should
% precede each author name, affiliation/snail-mail address and
% e-mail address. Additionally, tag each line of
% affiliation/address with \affaddr, and tag the
% e-mail address with \email.
%
% 1st. author
\alignauthor
  Richard Littauer\\
  \email{richard.littauer@gmail.com}
}

\maketitle
\begin{abstract}
Beagle is a client-side application that facilitates both annotation on PDFs and web
pages and traversing open source academic research. Structurally, Beagle is
envisioned as a browserified node application based on a peer-to-peer distributed
file system, accessed through a browser extension and a stand-alone application.
The UI is envisioned primarily as inline annotations combined with an informative
sidebar overlayed on the PDF or wbe page. The application is inherently modular,
which serves the dual purpose of maximising extensibility for the code base
(which could then be utilized by other members of the open research community)
and allowing users to select what information they can see. This document outlines
the technical specifications underlying Beagle.
\end{abstract}

\section{Introduction}

[Motivate Beagle. Introduce problems. Open Research. Citation graph. Competitors.
Partners. Chrome Extension. Data storage. User interaction. Encrytion. Future ]

\section{Front Ends}
\subsection{Chrome Extension}

Beagle is mainly realised as a Chrome extension. The extension loads a browserify
bundled javascript file. This file is the concatination of the the node.js
modules which are specified by the user; it also contains the React templates used
to automatically generate a siderbar and various modals which make up the UI for
the extension.

\subsubsection{Components}
\begin{itemize}
  \item Author Profile
  \item Publication Graph
  \item Tags
  \item Citation
  \item Saved papers
  \item Annotations
  \item Journal information
  \item Paper information
  \item Notes
\end{itemize}

\subsection{Standalone App}

Beagle will be able to pass on PDFs and web snippets via email to other users.
It will also be able to store PDFs and display them to users who share them with
other users.  % Possibly a copyright violation?



\section{Data Storage}

Beagle will store publicly citation data; data about viewed articles; data about authors;
publication graph data. It will also have a node available for accessing and storing
encrypted user data and annotations in a peer-to-peer distributed file system,
allowing for maximal uptime and decentralization of storage costs. Users can share
directly via their own node with other uses, bypassing the central storage, if
they wish, although all data sent will be hashed.

Some data will be optionally encrypted. For instance, tags on paper will have
the option of being public, semi-private (available to user groups), or private.
This allows Beagle to use tags to deduce contents of papers, or related papers.

% Open questions: Can we store pdf data if users upload it or use our system?
% If we can, that makes NLP for paper comparison much easier.

Encrypted data stored on a p2p network instead of in a data silo has several
distinct advantages; most namely, that the user is not dependant on a silo for
continued access to their data. The ensures the uptime of Beagle, regardless of
whether it is under active development or maintainence.

\section{User Sign On}

Users will sign on using 2 factor authentication; either through their email, or
through their phones. Their accounts will be linked to their email accounts; when
they log in, they will look for the new token in their email accounts, which can
be used on a one-time basis. Otherwise, their data is stored in their cookies
and in their browser.


\section{Acknowledgements}

This work has been supported by a MetaKnowledge Grant and by MIT.

%\bibliographystyle{abbrv}
%\bibliography{gfs}
%\balancecolumns
%\subsection{References}
\end{document}
