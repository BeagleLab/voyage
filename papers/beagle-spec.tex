\documentclass{sig-alternate}

\usepackage{tikz}
\usetikzlibrary{arrows}
\usetikzlibrary{trees}
\usetikzlibrary{positioning}

\usepackage{array}
\usepackage{amstext}
\usepackage{mathtools}
\DeclarePairedDelimiter{\ceil}{\lceil}{\rceil}

\begin{document}

\title{Beagle - Navigating Academic Research}
\subtitle{}

\numberofauthors{1}

\author{
% You can go ahead and credit any number of authors here,
% e.g. one 'row of three' or two rows (consisting of one row of three
% and a second row of one, two or three).
%
% The command \alignauthor (no curly braces needed) should
% precede each author name, affiliation/snail-mail address and
% e-mail address. Additionally, tag each line of
% affiliation/address with \affaddr, and tag the
% e-mail address with \email.
%
% 1st. author
\alignauthor
  Richard Littauer\\
  \email{richard.littauer@gmail.com}
}

\maketitle
\begin{abstract}
Beagle is a client-side application that facilitates both annotation on PDFs and web
pages and traversing open source academic research. Structurally, Beagle is
envisioned as a browserified node application based on a peer-to-peer distributed
file system, accessed through a browser extension and a stand-alone application.
The UI is envisioned primarily as inline annotations combined with an informative
sidebar overlayed on the PDF or wbe page. The application is inherently modular,
which serves the dual purpose of maximising extensibility for the code base
(which could then be utilized by other members of the open research community)
and allowing users to select what information they can see. This document outlines
the technical specifications underlying Beagle.
\end{abstract}

\section{Introduction}

[Motivate Beagle. Introduce problems. Open Research. Citation graph. Competitors.
Partners. Chrome Extension. Data storage. User interaction. Encrytion. Future ]

\section{Acknowledgements}

This work has been supported by a MetaKnowledge Grant and by MIT.

%\bibliographystyle{abbrv}
%\bibliography{gfs}
%\balancecolumns
%\subsection{References}
\end{document}
